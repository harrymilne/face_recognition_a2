\documentclass[a4paper]{article}
\usepackage[utf8]{inputenc}
\usepackage{fancyhdr}
\usepackage{geometry}
\usepackage{listings}
\usepackage{graphicx}
\usepackage{algorithmic}
\usepackage{float}
\usepackage{url}
\usepackage{hyperref}


\lstset{
    breaklines=true,
    numbers=left,
    frame=lines
}

\pagestyle{fancyplain}

\title{Evaluation}
\author{Harry Milne}
\date{March 2014}

\lhead{Harry Milne}
\chead{Candidate Number: 2677}
\rhead{Centre Number: 22169}

\cfoot{}
\rfoot{\thepage}

\begin{document}
\maketitle
\tableofcontents
\newpage
\section{Customer Requirements}
	
	\subsection{General Objectives}
	
		\begin{itemize}
			\item Quick and easy to use method of user verification

			The program allows an easy method to toggle the alarm system successfully, all the user has
			to do is let the program scan their face, therefore this objective has been met. (Line 8 in Figure~\ref{lst:interview})


			\item Easy additional user set-up

			Setting up users with the client currently requires a function within the code to be ran,
			this is within the ability of the user but it is hard to call it `easy'. However, the database
			has a user interface which works correctly and adding users is very easy. (Line 10 in Figure~\ref{lst:interview})

			\item Communication with the server over the network

			The client correctly communicates back and forth with the server across a network. (Line 12 in Figure~\ref{lst:interview})
 
			\item Store users and access logs in a SQL database

			Users and logs are stored on the server in an SQL database, this database is only controlled
			by the system administrator. (Line 14 in Figure~\ref{lst:interview})
		\end{itemize}

	\subsection{Specific Objectives}

		\paragraph{User Verification}
			\begin{itemize}
				\item Ability to recognize faces

				The client successfully detects that a face is within the camera region, the `SimpleCV' package
				handles this very well. (Line 16 in Figure~\ref{lst:interview})

				\item Ability to tell different faces from others
				
				After matching the input with the saved files, the client can correctly match an input face with a
				user that has previously been stored on the file system. (Line 18 in Figure~\ref{lst:interview})

				\item Must be able to match a name to that face

				Images of the users are saved to the system with the file name as the user's name, this means there is
				no further processing to find the user's name after the input has been matched with a file. (Line 18 in Figure~\ref{lst:interview})
			\end{itemize}

		\paragraph{User Set-up}

			\begin{itemize}
				\item UI to edit each user or add users

				A database UI has been created for the server, however there is no user interface for the client, so I would
				say this objective is partially met.
				
				\item Power user tools

				The set-up allows raw access to the SQLite database, however, there are no power user tools coded into any of
				the system.
			\end{itemize}

		\paragraph{Network Communication}
		
			\begin{itemize}
				\item Access to local area network
                    
                    This objective has been met because the server and client sucessfully talk over the local area network
                    via sockets; the client sending commands to the server and recieving a response. (Line 12 in Figure~\ref{lst:interview})

				\item Receive commands and send commands
                    
                    As mentioned in the above objective, commands work from remote hosts. When the client asks the `status' of
                    alarms, the server responds. Similarly, the server responds when the client wants to register a user 
                    accessing the system. (Line 12 in Figure~\ref{lst:interview})

				\item Secure
                    
                    The system isn't secure enough for anyone with a certain amount of computing knowledge, as it is possible to
                    just send commands the the server without being authenticated, meaning that if anyone sends a packet to the
                    correct port on the server, they can turn the system on or off.

			\end{itemize}

		\paragraph{SQL Database}
		
			\begin{itemize}
				\item Table containing each user

					The Users table contains the user\_id and user\_name attributes.

				\item Access logs with a relationship to a user,

					Each access entry contains a user\_id foreign key which points to the user who triggered that event.

				\item Secure interface with communicating to the database.

					The system isn't secure, it has merely coded to be functional. This is because of a time limitation set by my teacher.

			\end{itemize}

	\subsection{Core Objectives}
		
		\begin{itemize}
			\item User verification via face recognition

				This objective has been met as shown in figure~\ref{lst:interview}.

			\item Interface with the server

				The client talks to the server via the local area network, therefore this objective has been met.

			\item Server to turn the alarms on and off

				This objective has been met as shown in figure~\ref{lst:interview} line 20.
		\end{itemize}

	\subsection{Other Objectives}
		
		\begin{itemize}
			\item Socket encryption

				Does not exist in the system.

			\item Control over what time slots people are allowed to turn alarms off

				Does not exist in the system.

			\item Ability to edit users from different computers

				Does not exist in the system.
		\end{itemize}




\section{Effectiveness}

	\subsection{Specific Objectives}

		\paragraph{User Verification}
			\begin{itemize}

				\item Must be able to match a name to that face

				This objective could have been carried out better, there is different way to implement user face recognition called HaarCascade. 
				Unfortunately, the time constraints didn't give me enough time to find a valid way of making adding a user to this method of face
				recognition easily. HaarCascade takes `training' this is hard to make into some form of user interface.


			\end{itemize}

		\paragraph{User Set-up}

			\begin{itemize}
				\item UI to edit each user or add users

				The database browser allows adding of users by a simple pop-up dialogue, the browser also enables an intuitive `double-click' method
				of editing existing entities in the table. (Line 22 in Figure~\ref{lst:interview})

			\end{itemize}


		\paragraph{SQL Database}
		
			\begin{itemize}
				\item Table containing each user

				The structure of the user database table was simple as it just gave each user an id which could be linked to the related entities.

			\end{itemize}

	\subsection{Core Objectives}
		
		\begin{itemize}
			\item User verification via face recognition

			The problem with this implementation is that it is easily spoofed, there is no `blink detection' to combat the imitation of the user. 

			\item Interface with the server

			The clients commands to the server aren't very effective because there is no source authentication, an ideal fix to this would be some form of
			initial authentication phase so that the server has the client's unique identifier.
		\end{itemize}


\section{Learnability}
	The interface of the database browser is similar to other software's user interfaces, this makes it easy to understand. The client expressed this in
	line 22 figure~\ref{lst:interview}.

	\begin{figure}
		\begin{center}
		\caption{A screenshot of the database browser}
		\label{fig:db_browser}
		\includegraphics[scale=0.5]{../shared_assets/screenshots/db_browser_eg.png}
		\end{center}
	\end{figure}

	Above, in figure~\ref{fig:db_browser}, you can see how the elements match common software interfaces. For example, the `menubar' contains a `file'
	menu which contains things like `New' and `Open'. The table is also similar to that of excel, you can click to edit. 


\section{Maintainability}

	The code in this project is very modular, the different sections of it have been abstracted into classes, not only this but the functions inside these
	classes are mostly easy to understand since each function has a simple task. However, the code lacks comments in it's current state, so at the individual
	line level, it may require some more research as to the modules referenced in the code. The code only uses local variables, this means understanding where
	data comes from is a lot easier to that of a program which takes use of global variables.

\section{Suggestions for Improvement}

	\begin{itemize}
		\item Improved accuracy of face recognition

			The issue with this is in it's current state it takes way too long for the system to take a picture which is sufficient to match with the stored
			images of the user, this is vital because it is detrimental to the main objective of the system; to make the user authentication easier.

		\item Secure sockets

			This is a less important issue as the local network will actually already have protection from the fact you need to be connected to it in the first place,
			however this isn't an assumption I would like to make. The data sent to and from the client and server \textit{should} be encrypted.

		\item Authenticated clients 

			This is linked to the system being secure, without any kind of authentication anybody can send data to the server and get a response or trigger an event, this
			is highly undesirable.

		\item Synced databases

			Keeping all of the clients and server updated with the same version of database is something which would require a \textit{lot} of work from the system administrator,
			this currently is an unacceptable amount. Ideally, whenever a client is made on an \textit{authenticated} device, it'd tell the server and the server would inform the
			rest of the clients.

	\end{itemize}	

\section{End User Evidence Appendix}

	\subsection{Interview}

		\begin{figure}
			\caption{Feedback interview}
			\label{lst:interview}
			\lstinputlisting{../shared_assets/eval_interview.log}
			\includegraphics[scale=0.7]{../shared_assets/pmilnesignature.png}
		\end{figure}

\end{document}
